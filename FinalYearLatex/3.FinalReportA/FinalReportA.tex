\documentclass{UoNMCHA}
\usepackage[authoryear]{natbib}
\usepackage{array,booktabs} % For nice tables
\usepackage{amsmath,amsfonts,amssymb} % For nice maths
\usepackage{color}
\usepackage{enumerate}
\usepackage{listings}
\usepackage{subfig}
\usepackage{hyperref}
\usepackage[parfill]{parskip}   % For replacing paragraph indenting with a newline instead

\input{_content/____setup.txt}

\begin{document}
	\input{_content/00_titlePage.txt}

	\vspace{-5mm}
	\section*{Abstract}
	\vspace{-3mm}
		\input{_content/01_abstract.txt}

	\vspace{-2mm}
	\section*{Acknowledgements}
	\vspace{-3mm}
		\input{_content/02_acknowledgements.txt}

	\newpage\tableofcontents
	
	\newpage\section{Introduction}
		\input{_content/10_introduction.txt}

	\newpage\section{Core Section}\label{sec:Core Section}
		\input{_content/20_coreSection.txt}


	\newpage\section{References and Citations}\label{sec:RefCite}
Bibliographies are generated by LaTeX when the \verb|bibliography| command is used at the end of the document---see the bottom of the template. The \verb|bibliography| command requires you to provide a \texttt{.bib} file that contains all of the reference used in the document. The format of the bibliography and citations can be specified by using the \verb|bibliographystyle| command.

Bibliography \texttt{.bib} files can be created with a reference manager such as \href{https://www.jabref.org/}{JabRef}. To include a reference in your document, use the \verb|cite| command with the appropriate reference key. The appropriate key can be found within your reference manager. Alternatively, most reference editors have a short-cut to copy the cite command with key directly. In JarbRef, select the reference you want and press 'ctrl+k' to coppy the cite command with the corresponding key.

\begin{figure}[ht]
    \begin{center}
        %\includegraphics[width=.8\linewidth]{Figures/jabrefeg}
        %\caption{The bibtex reference key can be viewed in JabRef.}
        %\label{fig:jabref}
    \end{center}
\end{figure}

Variations on the \verb|cite| command are \verb|\citet{}| and \verb|\citep{}|---\cite{kajita_2002_a}, \citep{kajita_2002_a}, \citet{kajita_2002_a}. You can combine more than one reference in a single citation \citep{kajita_2002_a, kajita_2002_a}.


	\newpage\section{Conclusion}\label{sec:Conclusion}
		\input{_content/X0_conclusion.txt}



\bibliography{FinalReportA.bib} % This is the .bib file where the bibliography database is stored
\bibliographystyle{harvard} %% https://www.mybib.com/#/projects/XRL93q/citations


\appendix
\newpage
\section{Example of a Table}\label{app:Table}
 %
\begin{table}[h!]
    \begin{center}\label{tab:MCHAProg}
        \caption{Bachelor of Engineering Mechatronics First Year Program}\label{tab:notation}
        {\footnotesize
            \begin{tabular}{c l l l|}
                \hline\hline \textbf{1st Year} & & \\
                Semester & {Course Code} & {Course Name} \\ \hline 
                1 & GENG1003 & Introduction to Procedural Programming \\
                1 & GENG1500 & Introduction to Professional Engineering\\
                1 & MATH1110 & Maths for Engineering, Science \& Technology 1
\\
                1 & PHYS1205 & Advanced Physics I\\ \hline
                2 & CIVL1100 & Fundamentals of Engineering Mechanics\\
                2 & ELEC1310 & Introduction to Electrical Engineering\\
                2 & MATH1120 & Maths for Engineering, Science \& Technology 2 \\
                2 & MECH1110 & Mechanical Drawing/CAD \& Workshop Practice\\ \hline
            \end{tabular}
        }
    \end{center}
\end{table}

\end{document}



%\lstinputlisting[
%    language=Matlab,
%    float=h,
%    numbers=left,
%    xleftmargin=1cm,
%    frame=shadowbox,
%    caption={Matlab serial communication example.\label{lst:matlabserial}},
%    morekeywords={try,catch}
%    ]{Code/serialtest.m}

%%%%%%%%%%%%%%%%%%%%%%%%%%%%%%%%%%%%%%%%%%%%%%%%%%%%%%%%%%%%%%